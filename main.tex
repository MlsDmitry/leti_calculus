\documentclass[14pt, a4paper]{report}

\usepackage[T2A,T1]{fontenc}
\usepackage[utf8]{inputenc}
\usepackage[english, russian]{babel}
\usepackage{amsmath}
\usepackage{amssymb}
\usepackage{titlesec}
\usepackage{import}
\usepackage{xifthen}
\usepackage{pdfpages}
\usepackage{transparent}
\usepackage{physics}
\usepackage{amsthm}

\newtheorem{definition}{Определение}

\newcommand{\incfig}[1]{%
    \def\svgwidth{\columnwidth}
    \import{./figures/}{#1.pdf_tex}
}
\let\oldforall\forall
\renewcommand{\forall}{\oldforall \, }

\let\oldexist\exists
\renewcommand{\exists}{\oldexist \: }

\newcommand\existu{\oldexist! \: }

\begin{document}

\chapter{Числа}%
\label{cha:numbers}

\begin{itemize}
    \item $\mathbb{N}$ натуральные
    \item $\mathbb{Z}$ целые
    \item $\mathbb{Q}$ рациональные
        \begin{align*}
            \mathbb{Q} = \{ \frac{m}{n}, m \in \mathbb{Z}, n \in \mathbb{Z} \}
        \end{align*}
    \item $\mathbb{R}$ вещественные
\end{itemize}


\chapter{Комплексные числа}

\section{Свойства действительных чисел}

\begin{align*}
    & \mathbb{R} \times \mathbb{R}  \\
    & (a, b) + (c, d) := (a + c, b + d) \\
    & (a, b) - (c, d) := (a - c, b - d) \\
    & (a, b) \cdot (c, d) := (a \cdot c - b \cdot d, ad + bc) \\
    & (a, b) : (c, d) := (\frac{ac + bd}{c^2 + d^2}, \frac{-ad+bc}{c^2+d^2}) \\
\end{align*}


\section{Свойства комплексных чисел}

\begin{center}
    $\mathbb{C}$ множество комплексных чисел
\end{center}

\begin{enumerate}
    \item Коммутативность $z_1 + z_2 = z_2 + z_1$
    \item Ассоциативность $(z_1 + z_2) + z_3 = z_1 + (z_2 + z_3)$
    \item 
        $
        \begin{aligned}[t]
            & \exists ! 0 \in C : z + 0 = z, &\\
            & \forall z \in C, 0 = (0, 0) &\\
        \end{aligned}
        $
    \item
        $
        \begin{aligned}[t]
            & \forall z \in C, \exists! u \in C : z + u = 0 &\\
            & z = (a, b); u = (-a, -b) &\\
        \end{aligned}
        $
    \item $z_1 \cdot z_2 = z_2 \cdot z_1$
    \item Ассоциативность умножения
        \begin{align*}
            (z_1 \cdot z_2) \cdot z_3 = z_1 \cdot (z_2 \cdot z_3)
        \end{align*}
    \item $\forall! 1 \in C : z \cdot 1 = z, \forall z \in C$
        \begin{align*}
            \begin{cases}
                ac - bd = a \\
                ad + bc = b
            \end{cases}
            \begin{cases}
                ac^2 - bd = ac \\
                ad^2 + bdc = bd
            \end{cases}
            \begin{cases}
            a(c^2 + d^2) = ac + bd \\
            d = \frac{ac - a}{b}
        \end{cases}
        \end{align*}
\end{enumerate}

\end{document}
